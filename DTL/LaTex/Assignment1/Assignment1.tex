\documentclass{article}
\title{Data Structures and Algorithms}
\author{Omkar Oak}
\date{24-11-22}
\begin{document}
\maketitle
\tableofcontents
\newpage
\section{What is Data Structures}
A data structure is not only used for organizing the data. It is also used for processing, retrieving, and storing data. There are different basic and advanced types of data structures that are used in almost every program or software system that has been developed. So we must have good knowledge about data structures.
\subsection{Linear Data Structures}
Data structure in which data elements are arranged sequentially or linearly, where each element is attached to its previous and next adjacent elements, is called a linear data structure. 
\subsubsection{Static Data Structure}
Static data structure has a fixed memory size. It is easier to access the elements in a static data structure. 
 \subsubsection{Dynamic Data Structure}
In dynamic data structure, the size is not fixed. It can be randomly updated during the runtime which may be considered efficient concerning the memory (space) complexity of the code. 
\subsection{Non Linear Data Structures}
Data structures where data elements are not placed sequentially or linearly are called non-linear data structures. In a non-linear data structure, we can’t traverse all the elements in a single run only.
\newpage
\section{Arrays} 
\begin{itemize}
\item An array is a collection of items stored at contiguous memory locations. The idea is to store multiple items of the same type together.
\item This makes it easier to calculate the position of each element by simply adding an offset to a base value, i.e., the memory location of the first element of the array (generally denoted by the name of the array).
\end{itemize}
\section{Linked List}
A linked list is a linear data structure,in which the elements are not stored at contiguous memory locations.
\end{document}
